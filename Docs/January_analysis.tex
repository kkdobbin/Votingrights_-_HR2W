% Options for packages loaded elsewhere
\PassOptionsToPackage{unicode}{hyperref}
\PassOptionsToPackage{hyphens}{url}
%
\documentclass[
]{article}
\usepackage{amsmath,amssymb}
\usepackage{iftex}
\ifPDFTeX
  \usepackage[T1]{fontenc}
  \usepackage[utf8]{inputenc}
  \usepackage{textcomp} % provide euro and other symbols
\else % if luatex or xetex
  \usepackage{unicode-math} % this also loads fontspec
  \defaultfontfeatures{Scale=MatchLowercase}
  \defaultfontfeatures[\rmfamily]{Ligatures=TeX,Scale=1}
\fi
\usepackage{lmodern}
\ifPDFTeX\else
  % xetex/luatex font selection
\fi
% Use upquote if available, for straight quotes in verbatim environments
\IfFileExists{upquote.sty}{\usepackage{upquote}}{}
\IfFileExists{microtype.sty}{% use microtype if available
  \usepackage[]{microtype}
  \UseMicrotypeSet[protrusion]{basicmath} % disable protrusion for tt fonts
}{}
\makeatletter
\@ifundefined{KOMAClassName}{% if non-KOMA class
  \IfFileExists{parskip.sty}{%
    \usepackage{parskip}
  }{% else
    \setlength{\parindent}{0pt}
    \setlength{\parskip}{6pt plus 2pt minus 1pt}}
}{% if KOMA class
  \KOMAoptions{parskip=half}}
\makeatother
\usepackage{xcolor}
\usepackage[margin=1in]{geometry}
\usepackage{longtable,booktabs,array}
\usepackage{calc} % for calculating minipage widths
% Correct order of tables after \paragraph or \subparagraph
\usepackage{etoolbox}
\makeatletter
\patchcmd\longtable{\par}{\if@noskipsec\mbox{}\fi\par}{}{}
\makeatother
% Allow footnotes in longtable head/foot
\IfFileExists{footnotehyper.sty}{\usepackage{footnotehyper}}{\usepackage{footnote}}
\makesavenoteenv{longtable}
\usepackage{graphicx}
\makeatletter
\def\maxwidth{\ifdim\Gin@nat@width>\linewidth\linewidth\else\Gin@nat@width\fi}
\def\maxheight{\ifdim\Gin@nat@height>\textheight\textheight\else\Gin@nat@height\fi}
\makeatother
% Scale images if necessary, so that they will not overflow the page
% margins by default, and it is still possible to overwrite the defaults
% using explicit options in \includegraphics[width, height, ...]{}
\setkeys{Gin}{width=\maxwidth,height=\maxheight,keepaspectratio}
% Set default figure placement to htbp
\makeatletter
\def\fps@figure{htbp}
\makeatother
\setlength{\emergencystretch}{3em} % prevent overfull lines
\providecommand{\tightlist}{%
  \setlength{\itemsep}{0pt}\setlength{\parskip}{0pt}}
\setcounter{secnumdepth}{-\maxdimen} % remove section numbering
\usepackage{booktabs}
\usepackage{longtable}
\usepackage{array}
\usepackage{multirow}
\usepackage{wrapfig}
\usepackage{float}
\usepackage{colortbl}
\usepackage{pdflscape}
\usepackage{tabu}
\usepackage{threeparttable}
\usepackage{threeparttablex}
\usepackage[normalem]{ulem}
\usepackage{makecell}
\usepackage{xcolor}
\ifLuaTeX
  \usepackage{selnolig}  % disable illegal ligatures
\fi
\IfFileExists{bookmark.sty}{\usepackage{bookmark}}{\usepackage{hyperref}}
\IfFileExists{xurl.sty}{\usepackage{xurl}}{} % add URL line breaks if available
\urlstyle{same}
\hypersetup{
  pdftitle={January 2024 draft results},
  pdfauthor={Kristin Dobbin},
  hidelinks,
  pdfcreator={LaTeX via pandoc}}

\title{January 2024 draft results}
\author{Kristin Dobbin}
\date{2024-1-26}

\begin{document}
\maketitle

\hypertarget{background-research-questions-and-hypotheses}{%
\subsection{Background, research questions and
hypotheses}\label{background-research-questions-and-hypotheses}}

California has 2967 Community Water Systems (CWS) (including CWS on
Tribal lands regulated directly by EPA Region 9). These systems operate
under a wide diversity of governance arrangements as detailed by Dobbin
\& Fencl (2021). Governance type, in turn, corresponds with a variety of
important institutional considerations including how decisions are made,
the structure (and even existence) of the governing board, and how the
governing board is selected and by who.

Recent years have seen a growth of scholarship related to the
fundamental importance of good water governance, but, in part due to a
lack of data, little research has considered these questions at the
water system scale. Nonetheless, high-profile cases like the poisoning
of thousands in Flint Michigan highlight the ongoing importance of
transparent, representative, and accountable local leadership for water
justice.

To begin addressing this gap, this paper aims to describe and quantify
diversity in arrangements for representing customers in the
decision-making of Community Water Systems (hereafter
``enfranchisement'') and explore any potential associations with system
performance (see discussion of performance measure below). Specifically,
we ask: 1) How are drinking water customers represented in
decision-making in non-ancillary Community Water Systems in California?;
and 2) Is customer enfranchisement associated with system performance?

Regarding the second question, we expect that greater customer
enfranchisement will be associated improved performance and that this
association will be strongest for performance measures that relate to
affordability and access to public funding.

\hypertarget{data-summary}{%
\subsection{Data Summary}\label{data-summary}}

Our initial sample is 2458 of the total 2967 Community Water Systems
(CWS) in California. This sample excludes the 108 CWS on Tribal lands
regulated directly by EPA region 9 as well as ancillary water systems
serving state, federal or county facilities (e.g.~prisons, county owned
housing), schools and privately owned ancillary facilities (retreat
centers, churches, packing houses, farmworker housing etc.). We exclude
the former due to their unique governance and regulatory nature. We
exclude the later due to the fact that such ancillary systems are not
governed by boards (these rationale's need improvement, technically MHPs
aren't governed by boards either generally).

\hypertarget{focal-independent-variable-enfranchisment}{%
\subsubsection{Focal Independent variable:
enfranchisment}\label{focal-independent-variable-enfranchisment}}

We use an independent variable called ``enfranchisement'' to test our
hypotheses. This variable is factor variable with three levels: None,
limited and full. None indicates that customers within the water system
do not elect decision-makers for the system. This category is comprised
entirely of privately owned water systems, both Mobile Home Parks and
Investor Owned Utilities. The limited category indicates that only a
subset of land-owning residents within the service area are able to
votre to elect decision-makers. This category includes both private and
public systems including Mutual Benefit systems (MWCs, Homeowners
associations), some California Water Districts and some Irrigation
Districts. Finally the category full indicates that all registered
voters in the system/district are able to vote to elect decision makers
(thus non-citizen are still excluded, this category includes county
board of supervisors run systems (county subsidiary districts) where
residents only directly elect one member of the board but other members
are still elected). Notably, in 18 cases, we were unable to classify
systems into one of these three categories based on available
information. In another 14 cases, mostly JPAs, board members are
appointed by member agencies rather than by residents but these members
themselves are often generally elected. We exclude these cases from our
analyses. Thus the final sample size and distribution is as follows:

\begin{table}

\caption{\label{tab:remove unknowns and appointed by member agencies and table}Summary of enfranchisement independent variable}
\centering
\begin{tabular}[t]{lll}
\toprule
Variable & N & Percent\\
\midrule
enfranchisement\_final & 2426 & \\
... Full & 1012 & 42\%\\
... Limited & 797 & 33\%\\
... None & 617 & 25\%\\
\bottomrule
\end{tabular}
\end{table}

\hypertarget{questions-for-feedback-any-additional-thoughts-or-concerns-on-how-this-variable-is-broken-down-any-objections-to-excluding-tribal-and-jpa-systems-is-enfrachisement-the-right-term-to-us-as-the-focal-topic}{%
\subparagraph{Questions for feedback: Any additional thoughts or
concerns on how this variable is broken down? Any objections to
excluding Tribal and JPA systems? Is enfrachisement the right term to us
as the focal
topic?}\label{questions-for-feedback-any-additional-thoughts-or-concerns-on-how-this-variable-is-broken-down-any-objections-to-excluding-tribal-and-jpa-systems-is-enfrachisement-the-right-term-to-us-as-the-focal-topic}}

\hypertarget{outcome-variables}{%
\subsubsection{Outcome variables}\label{outcome-variables}}

System performance can be measured in a variety of ways and there is
reason to believe that certain elements of performance may be more
influenced by accountability and representation that others. For this
reason I propose to use several distinct measures of performance: 1)
whether a water system is on California's failing systems list (binary
variable, Y/N); 2) System Needs Assessment risk score (numeric variable,
can be disagregated to look at the needs assessment components
individually: water quality, accessibility, affordability and TMF risk);
and 3) whether the water system applied for COVID-19 arrearage relief on
behalf of their customers. In addition to or instead of the COVID
arrearage relief, we also look at Funding received since 2017. These
outcome measures are summarized below.

\begin{longtable}[]{@{}lrr@{}}
\toprule\noalign{}
& Count & Percent of all systems \\
\midrule\noalign{}
\endhead
\bottomrule\noalign{}
\endlastfoot
Failing & 281 & 0.12 \\
Not Failing & 2141 & 0.88 \\
NA's & 4 & 0.00 \\
\end{longtable}

\begin{table}

\caption{\label{tab:tabular overview of DVs}Summary statistics for risk score and components}
\centering
\begin{tabular}[t]{llllllll}
\toprule
Variable & N & Mean & Std. Dev. & Min & Pctl. 25 & Pctl. 75 & Max\\
\midrule
Total\_risk & 2278 & 2.2 & 1.5 & 0 & 1 & 3 & 10\\
WEIGHTED\_WATER\_QUALITY\_SCORE & 2278 & 0.53 & 0.86 & 0 & 0 & 0.75 & 4.5\\
WEIGHTED\_ACCESSIBILITY\_SCORE & 2278 & 0.81 & 0.73 & 0 & 0.33 & 1.3 & 4\\
WEIGHTED\_AFFORDABILITY\_SCORE & 2277 & 0.51 & 0.56 & 0 & 0 & 0.67 & 2\\
WEIGHTED\_TMF\_CAPACITY\_SCORE & 2278 & 0.36 & 0.41 & 0 & 0 & 0.67 & 2.8\\
\bottomrule
\end{tabular}
\end{table}

\begin{table}

\caption{\label{tab:tabular overview of DVs}Summary of arrearage applications}
\centering
\begin{tabular}[t]{lll}
\toprule
Variable & N & Percent\\
\midrule
Application.complete. & 2408 & \\
... No & 1542 & 64\%\\
... Yes & 866 & 36\%\\
\bottomrule
\end{tabular}
\end{table}

\begin{table}

\caption{\label{tab:tabular overview of DVs}Summary of funding received since 2017}
\centering
\begin{tabular}[t]{llllllll}
\toprule
Variable & N & Mean & Std. Dev. & Min & Pctl. 25 & Pctl. 75 & Max\\
\midrule
FUNDING\_RECEIVED\_SINCE\_2017 & 2422 & 1314849 & 18617681 & 0 & 0 & 0 & 697479677\\
\bottomrule
\end{tabular}
\end{table}

\hypertarget{questions-for-feedback-do-these-performance-measures-make-sense-are-the-comprehensive-enough-for-elements-of-both-how-you-can-think-about-system-success-and-also-diverse-enough-across-considerations-that-could-be-influenced-by-enfrachisement}{%
\subparagraph{Questions for feedback: Do these performance measures make
sense? Are the comprehensive (enough) for elements of both how you can
think about system ``success'' and also diverse enough across
considerations that could be influenced by
enfrachisement?}\label{questions-for-feedback-do-these-performance-measures-make-sense-are-the-comprehensive-enough-for-elements-of-both-how-you-can-think-about-system-success-and-also-diverse-enough-across-considerations-that-could-be-influenced-by-enfrachisement}}

\hypertarget{controlscovariates}{%
\subsubsection{Controls/covariates}\label{controlscovariates}}

To keep things simple, I'm thinking of using three controls/covariates
in the analyses: 1) population served (log transformed); 2) water source
(groundwater or surface water); and 3) whether water is purchased or
self-produced.

\hypertarget{questions-for-feedback-any-other-controls-that-should-be-considered}{%
\subparagraph{Questions for feedback: Any other controls that should be
considered?}\label{questions-for-feedback-any-other-controls-that-should-be-considered}}

\hypertarget{initial-analysis-for-the-relationship-between-enfrachisement-and-each-outcome-measure}{%
\subsection{Initial analysis for the relationship between enfrachisement
and each outcome
measure}\label{initial-analysis-for-the-relationship-between-enfrachisement-and-each-outcome-measure}}

\begin{verbatim}
## 
## Call:
## glm(formula = CURRENT_FAILING ~ enfranchisement_final + LN_POP + 
##     Source + Purchased, family = binomial, data = Data)
## 
## Deviance Residuals: 
##     Min       1Q   Median       3Q      Max  
## -0.7301  -0.5622  -0.5159  -0.2119   2.9206  
## 
## Coefficients:
##                              Estimate Std. Error z value Pr(>|z|)    
## (Intercept)                  -3.21416    0.48384  -6.643 3.07e-11 ***
## enfranchisement_finalFull     0.22668    0.17801   1.273  0.20288    
## enfranchisement_finalLimited  0.12072    0.16498   0.732  0.46432    
## LN_POP                       -0.08956    0.03417  -2.621  0.00878 ** 
## SourceSW                     -0.24714    0.19321  -1.279  0.20086    
## PurchasedSelf-produced        1.80143    0.41111   4.382 1.18e-05 ***
## ---
## Signif. codes:  0 '***' 0.001 '**' 0.01 '*' 0.05 '.' 0.1 ' ' 1
## 
## (Dispersion parameter for binomial family taken to be 1)
## 
##     Null deviance: 1738.4  on 2420  degrees of freedom
## Residual deviance: 1665.1  on 2415  degrees of freedom
##   (5 observations deleted due to missingness)
## AIC: 1677.1
## 
## Number of Fisher Scoring iterations: 6
\end{verbatim}

\includegraphics{/Users/KristinDobbin/Box Sync/Projects/R_Projects/Votingrights_-_HR2W/Docs/January_analysis_files/figure-latex/analyses-1.pdf}

\begin{verbatim}
## 
## Call:
## lm(formula = Total_risk ~ enfranchisement_final + LN_POP + Source + 
##     Purchased, data = Data)
## 
## Residuals:
##     Min      1Q  Median      3Q     Max 
## -2.6672 -0.9762 -0.2729  0.7404  7.4771 
## 
## Coefficients:
##                              Estimate Std. Error t value Pr(>|t|)    
## (Intercept)                   3.10160    0.16573  18.715  < 2e-16 ***
## enfranchisement_finalFull     0.12783    0.08103   1.578    0.115    
## enfranchisement_finalLimited -0.34831    0.07602  -4.582 4.86e-06 ***
## LN_POP                       -0.22646    0.01777 -12.743  < 2e-16 ***
## SourceSW                      0.02245    0.08686   0.259    0.796    
## PurchasedSelf-produced        0.72548    0.10366   6.998 3.39e-12 ***
## ---
## Signif. codes:  0 '***' 0.001 '**' 0.01 '*' 0.05 '.' 0.1 ' ' 1
## 
## Residual standard error: 1.387 on 2271 degrees of freedom
##   (149 observations deleted due to missingness)
## Multiple R-squared:  0.1491, Adjusted R-squared:  0.1472 
## F-statistic: 79.59 on 5 and 2271 DF,  p-value: < 2.2e-16
\end{verbatim}

\includegraphics{/Users/KristinDobbin/Box Sync/Projects/R_Projects/Votingrights_-_HR2W/Docs/January_analysis_files/figure-latex/analyses-2.pdf}

\begin{verbatim}
## 
## Call:
## lm(formula = WEIGHTED_WATER_QUALITY_SCORE ~ enfranchisement_final + 
##     LN_POP + Source + Purchased, data = Data)
## 
## Residuals:
##     Min      1Q  Median      3Q     Max 
## -0.6427 -0.6111 -0.3624  0.1445  3.8971 
## 
## Coefficients:
##                               Estimate Std. Error t value Pr(>|t|)    
## (Intercept)                   0.350921   0.100818   3.481 0.000509 ***
## enfranchisement_finalFull    -0.007829   0.049292  -0.159 0.873814    
## enfranchisement_finalLimited -0.005847   0.046245  -0.126 0.899403    
## LN_POP                        0.004333   0.010811   0.401 0.688586    
## SourceSW                     -0.183187   0.052839  -3.467 0.000536 ***
## PurchasedSelf-produced        0.243678   0.063062   3.864 0.000115 ***
## ---
## Signif. codes:  0 '***' 0.001 '**' 0.01 '*' 0.05 '.' 0.1 ' ' 1
## 
## Residual standard error: 0.8437 on 2271 degrees of freedom
##   (149 observations deleted due to missingness)
## Multiple R-squared:  0.02921,    Adjusted R-squared:  0.02707 
## F-statistic: 13.67 on 5 and 2271 DF,  p-value: 3.608e-13
\end{verbatim}

\includegraphics{/Users/KristinDobbin/Box Sync/Projects/R_Projects/Votingrights_-_HR2W/Docs/January_analysis_files/figure-latex/analyses-3.pdf}

\begin{verbatim}
## 
## Call:
## lm(formula = WEIGHTED_TMF_CAPACITY_SCORE ~ enfranchisement_final + 
##     LN_POP + Source + Purchased, data = Data)
## 
## Residuals:
##     Min      1Q  Median      3Q     Max 
## -0.5605 -0.3207 -0.1059  0.3058  2.3452 
## 
## Coefficients:
##                               Estimate Std. Error t value Pr(>|t|)    
## (Intercept)                   0.640323   0.047636  13.442  < 2e-16 ***
## enfranchisement_finalFull    -0.069740   0.023291  -2.994  0.00278 ** 
## enfranchisement_finalLimited -0.192603   0.021851  -8.815  < 2e-16 ***
## LN_POP                       -0.033583   0.005108  -6.574 6.05e-11 ***
## SourceSW                      0.012775   0.024966   0.512  0.60893    
## PurchasedSelf-produced        0.029588   0.029797   0.993  0.32082    
## ---
## Signif. codes:  0 '***' 0.001 '**' 0.01 '*' 0.05 '.' 0.1 ' ' 1
## 
## Residual standard error: 0.3986 on 2271 degrees of freedom
##   (149 observations deleted due to missingness)
## Multiple R-squared:  0.05327,    Adjusted R-squared:  0.05118 
## F-statistic: 25.56 on 5 and 2271 DF,  p-value: < 2.2e-16
\end{verbatim}

\includegraphics{/Users/KristinDobbin/Box Sync/Projects/R_Projects/Votingrights_-_HR2W/Docs/January_analysis_files/figure-latex/analyses-4.pdf}

\begin{verbatim}
## 
## Call:
## lm(formula = WEIGHTED_AFFORDABILITY_SCORE ~ enfranchisement_final + 
##     LN_POP + Source + Purchased, data = Data)
## 
## Residuals:
##     Min      1Q  Median      3Q     Max 
## -0.6621 -0.4787 -0.1707  0.2927  1.7771 
## 
## Coefficients:
##                               Estimate Std. Error t value Pr(>|t|)    
## (Intercept)                   0.429296   0.066531   6.453 1.34e-10 ***
## enfranchisement_finalFull     0.100849   0.032521   3.101  0.00195 ** 
## enfranchisement_finalLimited  0.095725   0.030522   3.136  0.00173 ** 
## LN_POP                       -0.022937   0.007133  -3.216  0.00132 ** 
## SourceSW                      0.041342   0.034855   1.186  0.23569    
## PurchasedSelf-produced        0.170484   0.041598   4.098 4.31e-05 ***
## ---
## Signif. codes:  0 '***' 0.001 '**' 0.01 '*' 0.05 '.' 0.1 ' ' 1
## 
## Residual standard error: 0.5565 on 2270 degrees of freedom
##   (150 observations deleted due to missingness)
## Multiple R-squared:  0.02373,    Adjusted R-squared:  0.02158 
## F-statistic: 11.03 on 5 and 2270 DF,  p-value: 1.615e-10
\end{verbatim}

\includegraphics{/Users/KristinDobbin/Box Sync/Projects/R_Projects/Votingrights_-_HR2W/Docs/January_analysis_files/figure-latex/analyses-5.pdf}

\begin{verbatim}
## 
## Call:
## lm(formula = WEIGHTED_ACCESSIBILITY_SCORE ~ enfranchisement_final + 
##     LN_POP + Source + Purchased, data = Data)
## 
## Residuals:
##      Min       1Q   Median       3Q      Max 
## -1.26347 -0.50545 -0.07448  0.36827  3.01544 
## 
## Coefficients:
##                               Estimate Std. Error t value Pr(>|t|)    
## (Intercept)                   1.682841   0.076741  21.929  < 2e-16 ***
## enfranchisement_finalFull     0.103970   0.037521   2.771  0.00563 ** 
## enfranchisement_finalLimited -0.246519   0.035201  -7.003 3.28e-12 ***
## LN_POP                       -0.174415   0.008229 -21.194  < 2e-16 ***
## SourceSW                      0.151492   0.040220   3.767  0.00017 ***
## PurchasedSelf-produced        0.281603   0.048002   5.867 5.10e-09 ***
## ---
## Signif. codes:  0 '***' 0.001 '**' 0.01 '*' 0.05 '.' 0.1 ' ' 1
## 
## Residual standard error: 0.6422 on 2271 degrees of freedom
##   (149 observations deleted due to missingness)
## Multiple R-squared:  0.2326, Adjusted R-squared:  0.2309 
## F-statistic: 137.7 on 5 and 2271 DF,  p-value: < 2.2e-16
\end{verbatim}

\includegraphics{/Users/KristinDobbin/Box Sync/Projects/R_Projects/Votingrights_-_HR2W/Docs/January_analysis_files/figure-latex/analyses-6.pdf}

\begin{verbatim}
## 
## Call:
## glm(formula = Application.complete. ~ enfranchisement_final + 
##     LN_POP + Source + Purchased, family = binomial, data = Data)
## 
## Deviance Residuals: 
##     Min       1Q   Median       3Q      Max  
## -2.4669  -0.6518  -0.3778   0.6004   2.7926  
## 
## Coefficients:
##                              Estimate Std. Error z value Pr(>|z|)    
## (Intercept)                  -3.76909    0.29089 -12.957  < 2e-16 ***
## enfranchisement_finalFull     0.32912    0.13507   2.437   0.0148 *  
## enfranchisement_finalLimited -0.96609    0.15898  -6.077 1.23e-09 ***
## LN_POP                        0.53445    0.03098  17.253  < 2e-16 ***
## SourceSW                     -0.02744    0.14623  -0.188   0.8512    
## PurchasedSelf-produced       -0.41139    0.17841  -2.306   0.0211 *  
## ---
## Signif. codes:  0 '***' 0.001 '**' 0.01 '*' 0.05 '.' 0.1 ' ' 1
## 
## (Dispersion parameter for binomial family taken to be 1)
## 
##     Null deviance: 3142.3  on 2403  degrees of freedom
## Residual deviance: 2127.4  on 2398  degrees of freedom
##   (22 observations deleted due to missingness)
## AIC: 2139.4
## 
## Number of Fisher Scoring iterations: 5
\end{verbatim}

\includegraphics{/Users/KristinDobbin/Box Sync/Projects/R_Projects/Votingrights_-_HR2W/Docs/January_analysis_files/figure-latex/analyses-7.pdf}

\begin{verbatim}
## 
## Call:
## lm(formula = FUNDING_RECEIVED_SINCE_2017 ~ enfranchisement_final + 
##     LN_POP + Source + Purchased, data = Data)
## 
## Residuals:
##       Min        1Q    Median        3Q       Max 
## -13106825  -2936675   -267019   1160777 682345254 
## 
## Coefficients:
##                               Estimate Std. Error t value Pr(>|t|)    
## (Intercept)                  -13228338    1948700  -6.788 1.42e-11 ***
## enfranchisement_finalFull       428896    1012480   0.424 0.671888    
## enfranchisement_finalLimited    926068     995342   0.930 0.352256    
## LN_POP                         1188220     188412   6.307 3.38e-10 ***
## SourceSW                       3633893    1074503   3.382 0.000731 ***
## PurchasedSelf-produced         6276512    1277276   4.914 9.52e-07 ***
## ---
## Signif. codes:  0 '***' 0.001 '**' 0.01 '*' 0.05 '.' 0.1 ' ' 1
## 
## Residual standard error: 18340000 on 2415 degrees of freedom
##   (5 observations deleted due to missingness)
## Multiple R-squared:  0.03157,    Adjusted R-squared:  0.02956 
## F-statistic: 15.74 on 5 and 2415 DF,  p-value: 2.786e-15
\end{verbatim}

\includegraphics{/Users/KristinDobbin/Box Sync/Projects/R_Projects/Votingrights_-_HR2W/Docs/January_analysis_files/figure-latex/analyses-8.pdf}

In many ways these results are as predicted, water quality and failing
system status don't offer many insights across enfranchisement types
which makes sense as we wouldn't necessarily expect enfranchisement to
influence water quality and water quality is a big factor for failing
status. The covid arrearage results align with the hypothesis.

Not sure how to interpret the overall risk score results -\textgreater{}
perhaps just confused by the different results related to the various
sub-components

A few other results are surprising. That limited enfrachisement systems
have better accessibility scores (less risk) is somewhat surprising. I
wonder if this relates to water rights/access to surface water?
Similarly that TMF risk scores are lowest among these systems is also
interesting and perhaps relates to greater resources among some larger
water districts and mutual water companies?

Below I dig a bit more some of the other unexpected results starting
with affordability. Data quality for the affordability component of the
needs assessment is notoriously challenging. A large reason for this is
that Mobile Home Park systems regularly don't charge customers for water
but rather incorporate the costs into rent. If we exclude MHPs from the
affordability analysis we find that the results are much more aligned
with initial expectations although limited enfrachisement systems
continue to stick out as being mroe resoruced/high performing.

\begin{verbatim}
## 
## Call:
## lm(formula = WEIGHTED_AFFORDABILITY_SCORE ~ enfranchisement_final + 
##     LN_POP + Source + Purchased, data = Data_noMHP)
## 
## Residuals:
##     Min      1Q  Median      3Q     Max 
## -0.7096 -0.4892 -0.2031  0.3885  1.7320 
## 
## Coefficients:
##                               Estimate Std. Error t value Pr(>|t|)    
## (Intercept)                   0.711016   0.085883   8.279 2.33e-16 ***
## enfranchisement_finalFull    -0.038514   0.044863  -0.858 0.390738    
## enfranchisement_finalLimited -0.088856   0.048368  -1.837 0.066356 .  
## LN_POP                       -0.039056   0.008001  -4.881 1.14e-06 ***
## SourceSW                      0.024419   0.037336   0.654 0.513158    
## PurchasedSelf-produced        0.158508   0.043581   3.637 0.000283 ***
## ---
## Signif. codes:  0 '***' 0.001 '**' 0.01 '*' 0.05 '.' 0.1 ' ' 1
## 
## Residual standard error: 0.5765 on 1879 degrees of freedom
##   (146 observations deleted due to missingness)
## Multiple R-squared:  0.0334, Adjusted R-squared:  0.03083 
## F-statistic: 12.98 on 5 and 1879 DF,  p-value: 1.915e-12
\end{verbatim}

\includegraphics{/Users/KristinDobbin/Box Sync/Projects/R_Projects/Votingrights_-_HR2W/Docs/January_analysis_files/figure-latex/additional analyses affordability-1.pdf}

Starting in 2023 the needs assessment also incorporated a new indicator
in the affordability risk score for household socioeconomic burden.
``The purpose of this risk indicator is to identify water systems that
serve communities that have both high levels of poverty and high housing
costs for low-income households. These communities may be struggling to
pay their current water bill and may have a difficult time shouldering
future customer charge increases when their limited disposable income is
constrained by high housing costs. This indicator is a composite
indicator of two data points: Poverty Prevalence and Housing Burden.''
As such, this indicator doesn't specifically relate to water rates or
affordability but rather social vulnerability. If we use one of the
other two subindicators instead of the affordability risk score
(maintaing use of the data set that does not include Mobile Home Parks)
we see similar results for the MHI indicator but for the extreme water
bill indicator we see limited enfrisement systems performing worst of
all followed by no enfrachisement.

\begin{verbatim}
## 
## Call:
## lm(formula = PERCENT_OF_MEDIAN_HOUSEHOLD_INCOME_MHI_RAW_SCORE ~ 
##     enfranchisement_final + LN_POP + Source + Purchased, data = Data_noMHP)
## 
## Residuals:
##     Min      1Q  Median      3Q     Max 
## -0.4589 -0.2758 -0.1494  0.3703  1.0287 
## 
## Coefficients:
##                               Estimate Std. Error t value Pr(>|t|)    
## (Intercept)                   0.527737   0.057033   9.253  < 2e-16 ***
## enfranchisement_finalFull    -0.053237   0.029317  -1.816 0.069557 .  
## enfranchisement_finalLimited -0.081867   0.031883  -2.568 0.010320 *  
## LN_POP                       -0.053403   0.005303 -10.070  < 2e-16 ***
## SourceSW                      0.079137   0.024737   3.199 0.001404 ** 
## PurchasedSelf-produced        0.107877   0.028985   3.722 0.000204 ***
## ---
## Signif. codes:  0 '***' 0.001 '**' 0.01 '*' 0.05 '.' 0.1 ' ' 1
## 
## Residual standard error: 0.3656 on 1707 degrees of freedom
##   (318 observations deleted due to missingness)
## Multiple R-squared:  0.09239,    Adjusted R-squared:  0.08973 
## F-statistic: 34.75 on 5 and 1707 DF,  p-value: < 2.2e-16
\end{verbatim}

\includegraphics{/Users/KristinDobbin/Box Sync/Projects/R_Projects/Votingrights_-_HR2W/Docs/January_analysis_files/figure-latex/additional analyses affordability 2-1.pdf}

\begin{verbatim}
## 
## Call:
## lm(formula = EXTREME_WATER_BILL_RAW_SCORE ~ enfranchisement_final + 
##     LN_POP + Source + Purchased, data = Data_noMHP)
## 
## Residuals:
##      Min       1Q   Median       3Q      Max 
## -0.35748 -0.17555 -0.09458  0.00931  1.02333 
## 
## Coefficients:
##                               Estimate Std. Error t value Pr(>|t|)    
## (Intercept)                   0.315156   0.045176   6.976 4.32e-12 ***
## enfranchisement_finalFull    -0.008771   0.023223  -0.378    0.706    
## enfranchisement_finalLimited  0.035086   0.025253   1.389    0.165    
## LN_POP                       -0.038033   0.004200  -9.055  < 2e-16 ***
## SourceSW                      0.097559   0.019595   4.979 7.04e-07 ***
## PurchasedSelf-produced        0.033590   0.022963   1.463    0.144    
## ---
## Signif. codes:  0 '***' 0.001 '**' 0.01 '*' 0.05 '.' 0.1 ' ' 1
## 
## Residual standard error: 0.2897 on 1710 degrees of freedom
##   (315 observations deleted due to missingness)
## Multiple R-squared:  0.09011,    Adjusted R-squared:  0.08744 
## F-statistic: 33.87 on 5 and 1710 DF,  p-value: < 2.2e-16
\end{verbatim}

\includegraphics{/Users/KristinDobbin/Box Sync/Projects/R_Projects/Votingrights_-_HR2W/Docs/January_analysis_files/figure-latex/additional analyses affordability 2-2.pdf}

\begin{verbatim}
## 
## Call:
## lm(formula = WEIGHTED_AFFORDABILITY_SCORE_NOSEB ~ enfranchisement_final + 
##     LN_POP + Source + Purchased, data = Data_noMHP)
## 
## Residuals:
##     Min      1Q  Median      3Q     Max 
## -0.8672 -0.4649 -0.2290  0.4607  2.0486 
## 
## Coefficients:
##                               Estimate Std. Error t value Pr(>|t|)    
## (Intercept)                   0.949854   0.097563   9.736  < 2e-16 ***
## enfranchisement_finalFull    -0.084492   0.050150  -1.685 0.092213 .  
## enfranchisement_finalLimited -0.105789   0.054540  -1.940 0.052586 .  
## LN_POP                       -0.099214   0.009072 -10.937  < 2e-16 ***
## SourceSW                      0.167933   0.042316   3.969 7.53e-05 ***
## PurchasedSelf-produced        0.178441   0.049582   3.599 0.000329 ***
## ---
## Signif. codes:  0 '***' 0.001 '**' 0.01 '*' 0.05 '.' 0.1 ' ' 1
## 
## Residual standard error: 0.6254 on 1707 degrees of freedom
##   (318 observations deleted due to missingness)
## Multiple R-squared:  0.107,  Adjusted R-squared:  0.1044 
## F-statistic:  40.9 on 5 and 1707 DF,  p-value: < 2.2e-16
\end{verbatim}

\includegraphics{/Users/KristinDobbin/Box Sync/Projects/R_Projects/Votingrights_-_HR2W/Docs/January_analysis_files/figure-latex/additional analyses affordability 2-3.pdf}

Considering the funding received results, it makes sense that systems
that are not failing and not risk would not necessarily be receiving
grant funding to the same degree as those that are. Thus below I re-do
this analysis but only failing water systems and then for failing and
at-risk system. Particularly for failing systems we do see a shift
towards full enfranchisement systems receiving more fundingalthough the
confidence intervals are large.

\begin{verbatim}
## 
## Call:
## lm(formula = FUNDING_RECEIVED_SINCE_2017 ~ enfranchisement_final + 
##     LN_POP + Source + Purchased, data = Data_failingonly)
## 
## Residuals:
##      Min       1Q   Median       3Q      Max 
## -4825338 -1129985  -285484   323316 26256228 
## 
## Coefficients:
##                              Estimate Std. Error t value Pr(>|t|)    
## (Intercept)                  -3974226    1599892  -2.484   0.0136 *  
## enfranchisement_finalFull      823026     552805   1.489   0.1377    
## enfranchisement_finalLimited    13625     481223   0.028   0.9774    
## LN_POP                         586075     131952   4.442  1.3e-05 ***
## SourceSW                      -372768     566843  -0.658   0.5113    
## PurchasedSelf-produced        1247543    1290762   0.967   0.3346    
## ---
## Signif. codes:  0 '***' 0.001 '**' 0.01 '*' 0.05 '.' 0.1 ' ' 1
## 
## Residual standard error: 3146000 on 275 degrees of freedom
## Multiple R-squared:  0.1428, Adjusted R-squared:  0.1272 
## F-statistic: 9.164 on 5 and 275 DF,  p-value: 4.441e-08
\end{verbatim}

\includegraphics{/Users/KristinDobbin/Box Sync/Projects/R_Projects/Votingrights_-_HR2W/Docs/January_analysis_files/figure-latex/additional analyses funding received-1.pdf}

\begin{verbatim}
## 
## Call:
## lm(formula = FUNDING_RECEIVED_SINCE_2017 ~ enfranchisement_final + 
##     LN_POP + Source + Purchased, data = Data_failingoratrisk)
## 
## Residuals:
##      Min       1Q   Median       3Q      Max 
## -4603668  -875904  -228695   328407 26430308 
## 
## Coefficients:
##                              Estimate Std. Error t value Pr(>|t|)    
## (Intercept)                  -4655564    1005690  -4.629 5.45e-06 ***
## enfranchisement_finalFull      178336     413494   0.431   0.6666    
## enfranchisement_finalLimited   130095     375394   0.347   0.7292    
## LN_POP                         662099     107828   6.140 2.58e-09 ***
## SourceSW                      -352271     412206  -0.855   0.3934    
## PurchasedSelf-produced        1650257     691022   2.388   0.0175 *  
## ---
## Signif. codes:  0 '***' 0.001 '**' 0.01 '*' 0.05 '.' 0.1 ' ' 1
## 
## Residual standard error: 2614000 on 303 degrees of freedom
## Multiple R-squared:  0.1651, Adjusted R-squared:  0.1513 
## F-statistic: 11.98 on 5 and 303 DF,  p-value: 1.345e-10
\end{verbatim}

\includegraphics{/Users/KristinDobbin/Box Sync/Projects/R_Projects/Votingrights_-_HR2W/Docs/January_analysis_files/figure-latex/additional analyses funding received-2.pdf}

Lastly I consider which Californians are served by systems with
different types of enfranchisement looking at the association between
enfranchisement and calenviroscreen scores and enfranchisement and the
socioeconomic burden indicators from the needs assessment. No
``enfranchisement'' systems are associated with higher cal enviro
scores. Limited enfrachisement systems, however, as associated with
lower scores. The results for socioeconomic burden are similar.

\begin{verbatim}
## 
## Call:
## lm(formula = CALENVIRO_SCREEN_SCORE ~ enfranchisement_final + 
##     LN_POP + Source + Purchased, data = Data)
## 
## Residuals:
##     Min      1Q  Median      3Q     Max 
## -29.116 -10.069  -3.164   7.255  57.727 
## 
## Coefficients:
##                              Estimate Std. Error t value Pr(>|t|)    
## (Intercept)                   25.7333     1.5154  16.981  < 2e-16 ***
## enfranchisement_finalFull     -1.3252     0.7931  -1.671   0.0949 .  
## enfranchisement_finalLimited  -4.6313     0.7838  -5.909 3.96e-09 ***
## LN_POP                         0.8789     0.1459   6.025 1.96e-09 ***
## SourceSW                      -7.0755     0.8334  -8.490  < 2e-16 ***
## PurchasedSelf-produced        -3.8670     0.9829  -3.934 8.59e-05 ***
## ---
## Signif. codes:  0 '***' 0.001 '**' 0.01 '*' 0.05 '.' 0.1 ' ' 1
## 
## Residual standard error: 13.88 on 2276 degrees of freedom
##   (144 observations deleted due to missingness)
## Multiple R-squared:  0.06249,    Adjusted R-squared:  0.06043 
## F-statistic: 30.34 on 5 and 2276 DF,  p-value: < 2.2e-16
\end{verbatim}

\includegraphics{/Users/KristinDobbin/Box Sync/Projects/R_Projects/Votingrights_-_HR2W/Docs/January_analysis_files/figure-latex/who is served-1.pdf}

\begin{verbatim}
## 
## Call:
## lm(formula = HOUSEHOLD_SOCIOECONOMIC_BURDEN_RAW_SCORE ~ enfranchisement_final + 
##     LN_POP + Source + Purchased, data = Data)
## 
## Residuals:
##      Min       1Q   Median       3Q      Max 
## -0.59950 -0.35907  0.01226  0.46713  0.79502 
## 
## Coefficients:
##                               Estimate Std. Error t value Pr(>|t|)    
## (Intercept)                   0.307016   0.048458   6.336 2.84e-10 ***
## enfranchisement_finalFull    -0.083344   0.023689  -3.518 0.000443 ***
## enfranchisement_finalLimited -0.138417   0.022250  -6.221 5.87e-10 ***
## LN_POP                        0.020423   0.005202   3.926 8.89e-05 ***
## SourceSW                     -0.074594   0.025377  -2.939 0.003322 ** 
## PurchasedSelf-produced        0.065835   0.030287   2.174 0.029833 *  
## ---
## Signif. codes:  0 '***' 0.001 '**' 0.01 '*' 0.05 '.' 0.1 ' ' 1
## 
## Residual standard error: 0.405 on 2261 degrees of freedom
##   (159 observations deleted due to missingness)
## Multiple R-squared:  0.03388,    Adjusted R-squared:  0.03175 
## F-statistic: 15.86 on 5 and 2261 DF,  p-value: 2.209e-15
\end{verbatim}

\includegraphics{/Users/KristinDobbin/Box Sync/Projects/R_Projects/Votingrights_-_HR2W/Docs/January_analysis_files/figure-latex/who is served-2.pdf}

\#\#\#\#QUESTIONS TO ANSWER LATER : 1) Use weighted or unweighted risk
assessment scores? Doesnt' matter for results.

\end{document}
